%%%%%%%%%%%%%%%%%%%%%%%%%%%%%%%%%%%%%%%%%
% Journal Article
% LaTeX Template
% Version 1.2 (15/5/13)
%
% This template has been downloaded from:
% http://www.LaTeXTemplates.com
%
% Original author:
% Frits Wenneker (http://www.howtotex.com)
%
% License:
% CC BY-NC-SA 3.0 (http://creativecommons.org/licenses/by-nc-sa/3.0/)
%
%%%%%%%%%%%%%%%%%%%%%%%%%%%%%%%%%%%%%%%%%

%----------------------------------------------------------------------------------------
%	PACKAGES AND OTHER DOCUMENT CONFIGURATIONS
%----------------------------------------------------------------------------------------

\documentclass[twoside]{article}

\usepackage[utf8]{inputenc}
\usepackage[ngerman]{babel}

\usepackage[sc]{mathpazo} % Use the Palatino font
\usepackage[T1]{fontenc} % Use 8-bit encoding that has 256 glyphs
\linespread{1.05} % Line spacing - Palatino needs more space between lines
\usepackage{microtype} % Slightly tweak font spacing for aesthetics

\usepackage[hmarginratio=1:1,top=32mm,columnsep=20pt]{geometry} % Document margins
\usepackage{multicol} % Used for the two-column layout of the document
\usepackage[hang, small,labelfont=bf,up,textfont=it,up]{caption} % Custom captions under/above floats in tables or figures
\usepackage{booktabs} % Horizontal rules in tables
\usepackage{float} % Required for tables and figures in the multi-column environment - they need to be placed in specific locations with the [H] (e.g. \begin{table}[H])
\usepackage{hyperref} % For hyperlinks in the PDF
\hypersetup{
  colorlinks   = true, %Colours links instead of ugly boxes
  urlcolor     = blue, %Colour for external hyperlinks
  linkcolor    = blue, %Colour of internal links
  citecolor   = red %Colour of citations
}

\usepackage{lettrine} % The lettrine is the first enlarged letter at the beginning of the text
\usepackage{paralist} % Used for the compactitem environment which makes bullet points with less space between them

\usepackage{apacite}

\usepackage{abstract} % Allows abstract customization
\renewcommand{\abstractnamefont}{\normalfont\bfseries} % Set the "Abstract" text to bold
\renewcommand{\abstracttextfont}{\normalfont\small\itshape} % Set the abstract itself to small italic text

\usepackage{titlesec} % Allows customization of titles
\renewcommand\thesection{\Roman{section}}
\titleformat{\section}[block]{\large\scshape\centering}{\thesection.}{1em}{} % Change the look of the section titles

\usepackage{graphicx}
\usepackage{pstricks}
\usepackage{pst-barcode}

\usepackage{fancyhdr} % Headers and footers
\pagestyle{fancy} % All pages have headers and footers
\fancyhead{} % Blank out the default header
\fancyfoot{} % Blank out the default footer
\fancyhead[C]{Scientifica Ambulo $\bullet$ Juni 2013 $\bullet$ Vol. I, No. 1} % Custom header text
\fancyfoot[RO,LE]{\thepage} % Custom footer text

%----------------------------------------------------------------------------------------
%	TITLE SECTION
%----------------------------------------------------------------------------------------

\title{\vspace{-15mm}\fontsize{24pt}{10pt}\selectfont\textbf{Die Ötschergräben}} % Article title

\author{
Michael Fladischer
\and
Martina Schlaipfer
\and
Matthias Schlaipfer
\vspace{-5mm}
}
\date{}

%----------------------------------------------------------------------------------------

\begin{document}

\maketitle % Insert title

\thispagestyle{fancy} % All pages have headers and footers

%----------------------------------------------------------------------------------------
%	ABSTRACT
%----------------------------------------------------------------------------------------

\begin{abstract}

\noindent Vereinzelt auch als der ``Grand Canyon Österreichs'' bezeichnet, liegen die Ötschergräben
südlich am Fuße des Ötscher-Gebirgsmassivs. Der sechs Kilometer lange Graben wird vom Ötscherbach durchzogen,
der auf ener Höhe von ca. 1100 Meter über dem Meeresspiegel entspringt. Bekannt für sein klares Wasser und
die Wasserfälle, die entlang des Ötscherbaches in den Graben stürzen, sind die Ötschergräben ein beliebtes
Ausflugsziel für Wanderer.

\end{abstract}

%----------------------------------------------------------------------------------------
%	ARTICLE CONTENTS
%----------------------------------------------------------------------------------------

\begin{multicols}{2} % Two-column layout throughout the main article text

\section{Einleitung}

\lettrine[nindent=0em,lines=3]{D} as Ötscher-Gebirgsmassiv wird geologisch dem Alpenvorland als Teil der
Kalkalpen zugerechnet. Zusammen mit dem Schneeberg und dem Wiener Becken bildet es die Kalkvoralpen. \cite{SH:2012}

%------------------------------------------------

\section{Ablauf}

%------------------------------------------------

\section{Teilnahme}

Die Abstimmung des Termins erfolgt über einen \href{http://www.doodle.com/m6ywebgyvg6632ig}{Doodle}. Smartphone-Benutzer
können den QR-code scannen um auf die entsprechende Seite zu gelangen.

\href{http://www.doodle.com/m6ywebgyvg6632ig}{
\pspicture(\columnwidth,\columnwidth)
\psbarcode{http://www.doodle.com/m6ywebgyvg6632ig}{height=2.8 width=2.8}{qrcode}
\endpspicture
}

Es stehen fünf Termine mit je drei Uhrzeiten zu Auswahl, wobei die Uhrzeit festlegt, wann wir uns am Parkplatz in wienerbruck/Erlaufklause treffen.

%------------------------------------------------

\section{Rückfragen}


%----------------------------------------------------------------------------------------
%	REFERENCE LIST
%----------------------------------------------------------------------------------------

\bibliographystyle{apacite}
\bibliography{literature}

%----------------------------------------------------------------------------------------

\end{multicols}

\end{document}
